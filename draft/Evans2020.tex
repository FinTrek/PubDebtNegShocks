\documentclass[letterpaper,12pt]{article}

\usepackage{threeparttable}
\usepackage{geometry}
\geometry{letterpaper,tmargin=1in,bmargin=1in,lmargin=1.25in,rmargin=1.25in}
\usepackage[format=hang,font=normalsize,labelfont=bf]{caption}
\usepackage{amsmath}
\usepackage{multirow}
\usepackage{array}
\usepackage{delarray}
\usepackage{amssymb}
\usepackage{amsthm}
\usepackage{lscape}
\usepackage{natbib}
\usepackage{setspace}
\usepackage{float,color}
\usepackage[pdftex]{graphicx}
\usepackage{pdfsync}
\usepackage{verbatim}
\usepackage{placeins}
\usepackage{geometry}
\usepackage{pdflscape}
\synctex=1
\usepackage{hyperref}
\hypersetup{colorlinks,linkcolor=red,urlcolor=blue,citecolor=red}
\usepackage{bm}
\usepackage{tikz}
\newcommand*\circled[1]{\tikz[baseline=(char.base)]{
            \node[shape=circle,draw,inner sep=1pt] (char) {#1};}}

\theoremstyle{definition}
\newtheorem{theorem}{Theorem}
\newtheorem{acknowledgement}[theorem]{Acknowledgement}
\newtheorem{algorithm}[theorem]{Algorithm}
\newtheorem{axiom}[theorem]{Axiom}
\newtheorem{case}[theorem]{Case}
\newtheorem{claim}[theorem]{Claim}
\newtheorem{conclusion}[theorem]{Conclusion}
\newtheorem{condition}[theorem]{Condition}
\newtheorem{conjecture}[theorem]{Conjecture}
\newtheorem{corollary}[theorem]{Corollary}
\newtheorem{criterion}[theorem]{Criterion}
\newtheorem{definition}{Definition} % Number definitions on their own
\newtheorem{derivation}{Derivation} % Number derivations on their own
\newtheorem{example}[theorem]{Example}
\newtheorem{exercise}[theorem]{Exercise}
\newtheorem{lemma}[theorem]{Lemma}
\newtheorem{notation}[theorem]{Notation}
\newtheorem{problem}[theorem]{Problem}
\newtheorem{proposition}{Proposition} % Number propositions on their own
\newtheorem{remark}[theorem]{Remark}
\newtheorem{solution}[theorem]{Solution}
\newtheorem{summary}[theorem]{Summary}
\bibliographystyle{aer}
\newcommand\ve{\varepsilon}
\renewcommand\theenumi{\roman{enumi}}
\newcommand\norm[1]{\left\lVert#1\right\rVert}
\newcommand\abs[1]{\left\lvert#1\right\rvert}

\begin{document}

\begin{titlepage}
\title{A Note on Negative Shocks, Public Debt, and Interest Rates \thanks{This research benefited from support from the \href{https://www.oselab.org/}{Open Source Economics Laboratory} at the University of Chicago. All Python code and documentation for the computational model is available at \href{https://github.com/OpenSourceEcon/PubDebtNegShocks}{https://github.com/OpenSourceEcon/PubDebtNegShocks}.}
}
\author{
  Richard W. Evans\footnote{University of Chicago, M.A. Program in Computational Social Science, McGiffert House, Room 208, Chicago, IL 60637, (773) 702-9169, \href{mailto:rwevans@uchicago.edu}{rwevans@uchicago.edu}.}
  }
\date{{\footnotesize{September 2019}} \\
  {\scriptsize{(version 19.09.a)}}}
\maketitle
\vspace{-9mm}
\begin{abstract}
  Put abstract here.
  \vspace{3mm}

  \noindent\textit{keywords:}\: [Put keywords here.]

  \vspace{3mm}

  \noindent\textit{JEL classification:} [Put JEL codes here.]

\end{abstract}
\thispagestyle{empty}
\end{titlepage}


\begin{spacing}{1.5}


\section{Introduction}\label{SecIntro}

  % \noindent Outline
  %    \begin{itemize}
  %       \item Fiscal limits have become important in the aftermath of the Great Recession as many countries are facing the risk of defaulting on their sovereign debt. In order to avoid default and reduce unfunded policies, governments are also changing revenue and expenditure characteristics of their fiscal transfer programs. These policy changes could be defined as a type of fiscal default in that the governments are reneging on promised transfers.
  %       \item \citet{LeeperWalker:2011} discuss ``models of fiscal limits and lays out a research agenda to integrate political economy and empirical considerations with general equilibrium models of monetary and fiscal interactions."
  %          \begin{itemize}
  %             \item They suggest that discussing large changes in fiscal policy, rather than sovereign debt default, is probably the more productive line to pursue for advanced economies.
  %             \item They define the fiscal limit as ``the point beyond which taxes and government expenditures can no longer adjust to stabilize the value of government debt."
  %          \end{itemize}
  %       \item Popular measures of a country's degree of indebtedness or fiscal insolvency, such as the deficit or the debt-to-GDP ratio, are not well defined or offer an incomplete picture. \citet{AuerbachGokhaleKotlikoff:1991} proposed a long-run measure of fiscal insolvency---``generational accounting''---in which the long-run fiscal burden is calculated as the net present value of expected future government revenues minus the net present value of expected future government expenditures minus the current debt. In the face of unsustainable policy, we create a measure of the degree to which promised government transfers are not sustainable in the long-run by the government's ability to pay. We define the fiscal gap as the net present value of promised transfers minus the net present value of expected actual transfers as a percent of the net present value of real output.
  %    \end{itemize}


\section{Economic Model}\label{SecModel}

  We study a simple 2-period-lived agent overlapping generations model in which the government promises to make a lump sum transfer $\bar{H}\geq 0$ from the young to the old each period. Ricardian equivalence holds in the sense that households have rational expectations and can forecast the effects of government budget imbalances. The constraints of the model generate states of the world in which the government can only make a transfer that is less that the promised amount $0 \leq H_t \leq \bar{H}$.

  Our characterization of government budget insolvency relies on the assumption that when the state of the world is such that $\bar{H}$ generates negative consumption for the young, the agents in the economy resort to autarky rather than starvation (negative consumption). This shut-down result would not hold if the government merely reduced the size of the transfer program in the face of a shut down. Rational agents would expect this and incorporate that risk on the payment $\bar{H}$ in the second period of their lives.\footnote{The argument here is that a proportional transfer program will never shut down a government. However, if the government is locked in to some degree of nonproportional transfer program, then there are states of the world in which the government must either shut down or default on that debt. If they default in a way that the consumption of the young does not go to zero, then the government has changed its nonproportional transfer program to look like a proportional transfer program.}


  \subsection{Household problem}\label{SecModelHH}

    A unit measure of identical consumer-worker households is born each period. A Household lives for exactly two periods indexed by $s=1,2$. They supply a unit of labor inelastically in both the young and old period of life $n_1>0$ and $n_2\geq 0$ in all periods $t$.

    In the first period of life, consumer-worker households choose how to divide their net labor income between age-$1$ consumption $c_{1,t}$ and capital investment (savings) with the firms $k_{2,t+1}$. The objective of a household is to maximize Epstein-Zin-Weil utility subject to a period budget constraint and two nonnegativity constraints.\footnote{See \citet{EpsteinZin:2013} and \citet{Weil:1990}.}
    \begin{align}
      &\max_{c_{1,t},k_{2,t+1},c_{2,t+1}}\: (1-\beta)\ln(c_{1,t}) + \beta\frac{1}{1-\gamma}\ln\Bigr(E_t\bigl[(c_{2,t+1})^{1-\gamma}\bigr]\Bigr) \quad \forall t \label{EqHHmaxUtil} \\
      &\quad\text{such that}\quad c_{1,t} + k_{2,t+1} = w_t n_1 + x_1 - H_t \label{EqHHbc1} \\
      &\quad\text{and}\quad c_{2,t+1} = \bigl(1+r_{t+1}\bigr)k_{2,t+1} + w_{t+1}n_2 + x_2 + H_{t+1} \label{EqHHbc2} \\
      &\quad\text{and}\quad c_{1,t},c_{2,t+1},k_{2,t+1} > 0 \label{EqHHnonneg}
    \end{align}
    The variables $w_t$ and $r_t$ are the wage and interest rate, respectively. And we follow \citet{Blanchard:2019} in including a potential exogenous transfer unrelated to the intergenerational transfer $H_t$ in the first period of life $x_1$ and in the second period of life $x_2$. These solvency preserving income transfers are one of the key drivers of the results from \cite{Blanchard:2019}.

    Let new households have no initial capital $k_{1,t} = 0$. Note that the nonnegativity constraints on consumption $c_{1,t}$ and $c_{2,t+1}$ are strict inequalities in equilibrium due to the Inada conditions on the period utility functions. Furthermore, we also do not allow the government transfer program to zero out the consumption and savings of the young, as shown in Section \ref{SecModelGovt} equation \eqref{EqGovt_Ht}. Finally, the strict inequality on savings $k_{2,t+1}>0$ is also an equilibrium condition that comes from the market clearing condition \eqref{EqModelMC_K}, that negative capital stock $K_t<0$ is not defined in the production function \eqref{EqModelFirmProdFunc}, and that zero capital stock $K_t=0$ would result in zero output $Y_t=0$ from \eqref{EqModelFirmProdFunc}, zero wage $w_t=0$ from \eqref{EqModelFirm_FOCL}, and an infinite interest rate $r_t=\infty$ from \eqref{EqModelFirm_FOCK}.

    Consumption in the second period of life is characterized by the second period budget constraint.
    \begin{equation}\tag{\ref{EqHHbc2}}
      c_{2,t+1} = (1+r_{t+1})k_{2,t+1} + w_{t+1}n_2 + x_2 + H_{t+1} \quad\forall t
    \end{equation}
    Note that the nonnegativity constraint on old-age consumption $c_{2,t+1}$ will never bind because everything on the right-hand-side of \eqref{EqHHbc2} is weakly positive. Consumption in the first period of life $c_{1,t}$ and savings in the first period of life $k_{2,t+1}$ are jointly determined by the first period budget constraint \eqref{EqHHbc1} and by the Euler equation that characterizes the optimal young $s=1$ consumption-savings decision that maximizes lifetime utility \eqref{EqHHmaxUtil} subject to constraints \eqref{EqHHbc1}, \eqref{EqHHbc2}, and \eqref{EqHHnonneg}.
    \begin{equation}\label{EqHHEul_c1}
      \frac{1-\beta}{c_{1,t}} = \beta \frac{E_t\left[(1 + r_{t+1})\bigl(c_{2,t+1}\bigr)^{-\gamma}\right]}{E_t\left[\bigl(c_{2,t+1}\bigr)^{1-\gamma}\right]} \quad\forall t
    \end{equation}

    By substituting the age $s=1$ and $s=2$ budget constraints \eqref{EqHHbc1} and \eqref{EqHHbc2} into the household Euler equation \eqref{EqHHEul_c1}, we can see that the characterizing equation for savings $k_{2,t+1}$ is one equation with one unknown.
    \begin{equation}\label{EqHHEul_k2}
      \frac{1-\beta}{w_t n_1 + x_1 - H_t - k_{2,t+1}} = \beta\frac{E_t\Bigl[(1 + r_{t+1})\bigl([1+r_{t+1}]k_{2,t+1} + w_{t+1}n_2 + x_2 + H_{t+1}\bigr)^{-\gamma}\Bigr]}{E_t\Bigl[\bigl([1+r_{t+1}]k_{2,t+1} + w_{t+1}n_2 + x_2 + H_{t+1}\bigr)^{1-\gamma}\Bigr]} \quad\forall t
    \end{equation}
    Equation \eqref{EqHHEul_k2} shows that the functional solution for household savings $k_{2,t+1}$ every period is a stationary function $\psi(\cdot)$ of the time path of transfers and prices over the lifetime of the household.
    \begin{equation}\label{EqHH_psi}
      k_{2,t+1} = \psi\bigl(H_t, H_{t+1}, w_t, w_{t+1}, r_{t+1}\bigr)
    \end{equation}


  \subsection{Firm problem}\label{SecModelFirm}

    A unit measure of identical perfectly competitive firms exist in this economy that hire aggregate labor $L_t$ at wage $w_t$ and rent aggregate capital $K_t$ at rental rate $r_t$ every period in order to produce consumption good $Y_t$ according to a Cobb-Douglas production function,
    \begin{equation}\label{EqModelFirmProdFunc}
       Y_t = F(K_t, L_t, z_t) = A_t\Bigl[(\alpha)^\frac{1}{\ve}(K_t)^\frac{\ve-1}{\ve} + (1 - \alpha)^\frac{1}{\ve}(L_t)^\frac{\ve-1}{\ve}\Bigr]^\frac{\ve}{\ve-1} \quad\forall t
    \end{equation}
    where the capital share of income is given by $\alpha\in(0,1)$ and $\ve>0$ is the constant elasticity of substitution between capital and labor in the production process. Total factor productivity $A_t = e^{z_t}>0$ is distributed log normally, and $z_t$ follows a normally distributed $AR(1)$ process.
    \begin{equation}\label{EqModelFirmZAR1}
      \begin{split}
        z_t &= \rho z_{t-1} + (1-\rho)\mu + \epsilon_t \\
        &\text{where}\quad \rho\in[0,1),\quad\mu\geq 0, \quad\text{and}\quad \epsilon_t \sim N(0,\sigma)
      \end{split}
    \end{equation}
    The firm's problem each period is to choose how much capital $K_t$ to rent and how much labor $L_t$ to hire in order to maximize profits,
    \begin{equation}\label{EqModelFirmProfMax}
      \max_{K_t, L_t}\:Pr_t = F(K_t, L_t, z_t) - w_t L_t - (r_t + \delta)K_t \quad\forall t
    \end{equation}
    where $\delta$ is the per-period depreciation rate of capital. Profit maximization implies that the real wage and real rental rate are determined by the standard first order conditions for the firm.
    \begin{gather}
      r_t = (A_t)^\frac{\ve-1}{\ve}\left[\alpha\frac{Y_t}{K_t}\right]^\frac{1}{\ve} - \delta \quad\forall t \label{EqModelFirm_FOCK} \\
      w_t = (A_t)^\frac{\ve-1}{\ve}\left[(1-\alpha)\frac{Y_t}{L_t}\right]^\frac{1}{\ve} \quad\forall t \label{EqModelFirm_FOCL}
    \end{gather}

    Because the interest rate $r_t$ in \eqref{EqModelFirm_FOCK} is not defined when the capital stock is zero $K_t=0$ and because the wage $w_t$ in \eqref{EqModelFirm_FOCL} is not defined when aggregate labor is zero $L_t=0$, we know that both values must be strictly positive $K_t, L_t>0$.


  \subsection{Government transfer program}\label{SecModelGovt}

    We model a simple balanced budget public transfer program that takes an amount from the young each period $H_t$ and gives that same amount to the old each period, as shown in the young and old budget constraints.
    \begin{gather}
      c_{1,t} + k_{2,t+1} = w_t n_1 + x_1 - H_t \quad\forall t \tag{\ref{EqHHbc1}} \\
      c_{2,t} = \bigl(1+r_t\bigr)k_{2,t} + w_t n_2 + x_2 + H_t \quad\forall t \tag{\ref{EqHHbc2}}
    \end{gather}
    In contrast to the way the old-age ($s=2$) budget constraint \eqref{EqHHbc2} is displayed in Section \ref{SecModelHH}, we show the budget constraints here for a young household and old household both in period $t$ (two separate individuals). The government budget is made up entirely of this transfer program, and the budget is always balanced because the government revenue taken from the young in period $H_t$ is always equal to the transfers to the old $H_t$ in all periods $t$.

    In most periods, the government promises that the transfer will be $\bar{H}\geq 0$. However, in the case that $\bar{H}>0$, there could exist states of the economy such that $\bar{H}\geq w_t n_1 + x_1$. In these cases, net labor income is less than or equal to zero, so the strict inequalities on $c_{1,t}$ and $k_{2,t+1}$ in \eqref{EqHHnonneg} must be violated. To avoid negative consumption, we require that the most the government can take from the young in any period is all their income up to some arbitrarily small minimum consumption $c_{min}>0$ and an arbitrarily small amount of savings $K_{min}>0$.
    \begin{equation}\label{EqGovt_Ht}
      \begin{split}
        H_t &\equiv
          \begin{cases}
            \bar{H} \qquad\qquad\qquad\qquad\qquad\:\:\:\,\text{if}\quad w_t n_1 \geq \bar{H} - x_1 + c_{min} + K_{min} \\
            w_t n_1 + x_1 - c_{min} - K_{min} \quad\text{if}\quad w_t n_1 < \bar{H} - x_1 + c_{min} + K_{min}
          \end{cases} \quad\forall t \\
          &= \min\left(\bar{H}, w_t n_1 + x_1 - c_{min} - K_{min}\right) \quad\forall t
      \end{split}
    \end{equation}
    This rule states that the government implements a balanced budget transfer program from the young to the old every period. And for $\bar{H}>0$, once the wage dips low enough, the government can no longer take $\bar{H}$ from the young. In this case, the government takes all that it can from the young $H_t = w_t n_1 + x_1 - c_{min} - K_{min}<\bar{H}$ and transfers that amount to the old. The young are left with consumption and savings equal to the minimum $c_{1,t}=c_{min}$ and $k_{2,t+1}=K_{min}$, and the economy shuts down and devolves into autarky.


  \subsection{Market clearing}\label{SecModelShutMktClr}

    Market clearing implies that the aggregate labor demand equals aggregate labor supply, aggregate capital demand equals aggregate capital supply, and output equals consumption plus investment in each period,
    \begin{align}
      L_t &= n_1 + n_2 \quad\forall t \label{EqModelMC_L} \\
      K_t &= k_{2,t} \quad\forall t \label{EqModelMC_K} \\
      \begin{split}
        Y_t &= C_t + K_{t+1} - (1-\delta)K_t \quad\forall t \\
        &\quad\text{where}\quad C_t\equiv c_{1,t} + c_{2,t}
      \end{split} \label{EqModelMC_rescnstr}
    \end{align}
    where the goods market clearing condition or resource constraint \eqref{EqModelMC_rescnstr} is redundant by Walras' Law.


  \subsection{Equilibrium}\label{SecModelEqlb}

    In this section, we define a functional stationary equilibrium in which our definition of stationary is that the functional forms are not time dependent. That is, for a function $f(\bm{x})$ of vector of variables $\bm{x}$, the function does not change. Only the output values of the function changes in response to changing inputs $\bm{x}$.

    \end{spacing}
    \vspace{5mm}
    \hrule
    \vspace{-1mm}
    \begin{definition}[\textbf{Functional stationary equilibrium}]\label{DefEqlb}
      A non-autarkic functional stationary equilibrium in the two-period-lived overlapping generations model with exogenous labor supply and aggregate shocks is defined by stationary price functions $r(k,z)$ and $w(k,z)$ and a stationary savings function $k'=\psi(k,z)$ for all current state wealth $k$ and total factor productivity component $z$ such that:
      \begin{enumerate}
        \item households optimize according to \eqref{EqHHbc1} and \eqref{EqHHbc2}, and \eqref{EqHHEul_c1}
        \item firms optimize according to \eqref{EqModelFirm_FOCK} and \eqref{EqModelFirm_FOCL},
        \item markets clear according to \eqref{EqModelMC_L} and \eqref{EqModelMC_K}.
      \end{enumerate}
    \end{definition}
    \vspace{-2mm}
    \hrule
    \vspace{5mm}
    \begin{spacing}{1.5}

    We can solve for the stationary price functions analytically by substituting the market clearing conditions \eqref{EqModelMC_L} and \eqref{EqModelMC_K} into the firms' respective first order conditions \eqref{EqModelFirm_FOCK} and \eqref{EqModelFirm_FOCL}.
    \begin{gather}
      r_t\equiv r\bigl(k_{2,t}, z_t\bigr) = (e^{z_t})^\frac{\ve-1}{\ve}\left[\alpha\frac{F(k_{2,t},n_1+n_2,z_t)}{k_{2,t}}\right]^\frac{1}{\ve} - \delta \quad\forall z_t \quad\text{and}\quad k_{2,t}>0 \label{EqModelEqlb_r} \\
      w_t\equiv w\bigl(k_{2,t}, z_t\bigr) = (e^{z_t})^\frac{\ve-1}{\ve}\left[(1-\alpha)\frac{F(k_{2,t},n_1+n_2,z_t)}{n_1+n_2}\right]^\frac{1}{\ve} \quad\forall z_t \quad\text{and}\quad k_{2,t}>0 \label{EqModelEqlb_w}
    \end{gather}
    We can also solve analytically for the equilibrium expression for the transfer each period $H_t$ as a function of the wealth of the current-period old $k_{2,t}$ and the value of the normally distributed component $z_t$ of the total factor productivity process (as well as the parameters of the promised transfer amount $\bar{H}$ and minimum values of young age consumption $c_{min}$ and aggregate capital $K_{min}$)  by substituting the equilibrium wage expression \eqref{EqModelEqlb_w} into the expression for $H_t$ \eqref{EqGovt_Ht}.
    \begin{equation}\label{EqModelEqlb_H}
      H_t \equiv H\bigl(k_{2,t}, z_t\bigr) = \min\Bigl(\bar{H}, w(k_{2,t},z_t)n_1 + x_1 - c_{min} - K_{min}\Bigr) \quad\forall z_t \quad\text{and}\quad k_{2,t}>0
    \end{equation}

    Finally, if we substitute the equilibrium expressions for prices $r(k,z)$ and $w(k,z)$ and the transfer $H(k,z)$ from \eqref{EqModelEqlb_r}, \eqref{EqModelEqlb_w}, and \eqref{EqModelEqlb_H} into the household Euler equation \eqref{EqHHEul_k2} and resulting policy function \eqref{EqHH_psi}, it is clear that the equilibrium savings function $k'=\psi(k,z)$ is a function of the wealth of the current-period old $k_{2,t}$ and the value $z_t$ of the normally distributed component of total factor productivity,
    \begin{equation}\label{EqModelEqlb_Eul}
      \begin{split}
        &\frac{1-\beta}{w(k_{2,t},z_t) n_1 + x_1 - H(k_{2,t},z_t) - k_{2,t+1}} = \\
        & \quad\beta\int_{z_{t+1}}\biggl[\bigl(1+r(k_{2,t+1},z_{t+1})\bigr)\times \\
        &\qquad\qquad\qquad \Bigl([1 + r(k_{2,t+1},z_{t+1})]k_{2,t+1} + w(k_{2,t+1},z_{t+1})n_2 + H(k_{2,t+1},z_{t+1})\Bigr)^{-\gamma}\times \\
        &\qquad\qquad\qquad\qquad\qquad f\bigl(z_{t+1}|\rho z_t + (1 - \rho)\mu, \sigma\bigr)\biggr]dz_{t+1} \:\div \\
        & \quad\int_{z_{t+1}}\biggl[\Bigl([1 + r(k_{2,t+1},z_{t+1})]k_{2,t+1} + w(k_{2,t+1},z_{t+1})n_2 + H(k_{2,t+1},z_{t+1})\Bigr)^{1-\gamma}\times \\
        &\qquad\qquad\qquad\qquad\qquad f\bigl(z_{t+1}|\rho z_t + (1 - \rho)\mu, \sigma\bigr)\biggr]dz_{t+1} \\
        &\qquad\qquad\qquad\qquad\qquad\qquad\qquad \forall z_t, z_{t+1} \quad\text{and}\quad k_{2,t}, k_{2,t+1}>0
      \end{split}
    \end{equation}
    \begin{equation}\label{EqModelEqlb_psi}
      k_{2,t+1} = \psi\bigl(k_{2,t},z_t\bigr)>0 \quad\forall z_t \quad\text{and}\quad k_{2,t}>0
    \end{equation}
    where $f(z_{t+1}|\rho z_t + (1-\rho)\mu,\sigma)$ is the probability density function of $z_{t+1}$ distributed normally with mean $\rho z_t + (1-\rho)\mu$ and standard deviation $\sigma$.


  \subsection{One-period riskless bonds}\label{SecModelRiskless}

    In this section, we derive the return on a riskless bond. We make the simplifying assumption that the riskless bonds are zero absolute supply. However, this characterization can be generalized to cases in which the riskless bonds have exogenous positive supply. Because of our zero-supply assumption on the riskless bond, we can separate its derivation from the characterization of the household problem in Section \ref{SecModelHH}. These zero-supply riskless bonds do not influence the rest of the economy. They simply represent another measure of the level of risk present in each period of the economy.

    Assume that households have two potential instruments for saving. A household can invest income with the production sector $k_{2,t+1}$ and earn a stochastic risky return next period of $r_t$ and they can buy $b_{2,t+1}$ units of a one-period riskless bond  for price $p_t$ that returns exactly $b_{2,t+1}$ when old. It is clear that old-age households will have no demand for these bonds.

    The maximization problem for a generic household can be characterized as choosing risky savings $k_{2,t+1}$ and riskless savings $b_{2,t+1}$ to maximize lifetime utility subject to budget constraints.
    \begin{align}
      &\max_{k_{2,t+1},b_{2,t+1}}\: (1-\beta)\ln(c_{1,t}) + \beta\frac{1}{1-\gamma}\ln\Bigl(E_t\bigl[(c_{2,t+1})^{1-\gamma}\bigr]\Bigr) \quad \forall t \label{EqModelRiskMaxUtil} \\
      &\quad\text{such that}\quad c_{1,t} + k_{2,t+1} + p_t b_{2,t+1} = w_t n_1 + x_1 - H_t \label{EqModelRisk_bc1} \\
      &\quad\text{and}\quad c_{2,t+1} = \bigl(1+r_{t+1}\bigr)k_{2,t+1} + b_{2,t+1} + w_{t+1}n_2 + x_2 + H_{t+1} \label{EqModelRisk_bc2} \\
      &\quad\text{and}\quad c_{1,t},c_{2,t+1},k_{2,t+1} \geq 0 \label{EqModelRisk_nonneg}
    \end{align}
    The optimality condition for risky savings $k_{2,t+1}$ is the same Euler equation as in Section \ref{SecModelHH}.
    \begin{equation}\tag{\ref{EqHHEul_c1}}
       \frac{1-\beta}{c_{1,t}} = \beta \frac{E_t\left[(1 + r_{t+1})\bigl(c_{2,t+1}\bigr)^{-\gamma}\right]}{E_t\left[\bigl(c_{2,t+1}\bigr)^{1-\gamma}\right]} \quad\forall t
    \end{equation}
    The Euler equation for riskless savings $b_{2,t+1}$ is the following,
    \begin{equation}\label{EqModelRisk_rbart}
      \begin{split}
        \frac{1}{1 + \bar{r}_t} \equiv p_t = \left(\frac{\beta}{1-\beta}\right)\frac{c_{1,t}E_t\Bigl[\bigl(c_{2,t+1}\bigr)^{-\gamma}\Bigr]}{E_t\Bigl[\bigl(c_{2,t+1}\bigr)^{1-\gamma}\Bigr]} \quad \forall t \\
        \Rightarrow\quad \bar{r}_t = \left(\frac{1-\beta}{\beta}\right)\frac{E_t\Bigl[\bigl(c_{2,t+1}\bigr)^{1-\gamma}\Bigr]}{c_{1,t}E_t\Bigl[\bigl(c_{2,t+1}\bigr)^{-\gamma}\Bigr]} - 1 \quad\forall t
      \end{split}
    \end{equation}
    where the price of the riskless bond $p_t$ is defined as the reciprocal of the gross riskless return $1 + \bar{r}_t$. The optimality conditions of the production sector are the same as in Section \ref{SecModelFirm}.

    Euler equation \eqref{EqModelRisk_rbart} determines the demand for riskless bonds. We assume, generally, an exogenous supply of riskless bonds that is nonnegative $B_t\geq 0$ for all $t$. However, specifically in this model, we assume a zero supply of riskless bonds $B_t=0$. So the general version of our riskless bond market clearing condition is the following.
    \begin{equation}\label{EqModelMC_B_gen}
      b_{2,t} = B_t \quad\forall t
    \end{equation}
    With our zero supply assumption $B_t = 0$, the household demand for riskless bonds is set to zero through the market clearing condition,
    \begin{equation}\label{EqModelMC_B_zero}
      b_{2,t} = 0 \quad\forall t
    \end{equation}
    all the other endogenous variables are determined by the equilibrium described in Section \ref{SecModelEqlb}, and the riskless return $\bar{r}_t$ is characterized by Euler equation \eqref{EqModelRisk_rbart}.

    If we were to relax our zero-supply assumption on riskless bonds $B_t>0$, we would have to determine the riskless return $\bar{r}_t$ jointly with the rest of the endogenous variables.

    And finally, because the agents in our model live for only two periods, it is intuitive that each model period must represent many years. If we assume that the average economic life is 60 years, then each model period represents 30 years. Let the parameter $yrs$ be the number of years represented in a model period. Then we can report the riskless interest rate $\bar{r}_t$ characterized in \eqref{EqModelRisk_rbart} as an annual rate $\bar{r}_{t,an}$ using the following expression.
    \begin{equation}\label{}
      \bar{r}_{t,an} = \bigl(1 + \bar{r}_t\bigr)^\frac{1}{yrs} - 1 \quad\forall t
    \end{equation}


\section{Simulations}\label{SecSims}

  We explore the properties of the model from Section \ref{SecModel} with respect to different values of the promised transfer $\bar{H}$, initial wealth $k_{2,0}$, and the extent and probability of low total factor productivity values $A_t$ by calibrating the other parameters of the model and simulating a time series of the model 3,000 times for different combinations of $\bar{H}$, $k_{2,0}$, and the support and distribution of $A_t$. The first three rows of Table \ref{TabCalibr} show the different values of $\bar{H}$, $k_{2,0}$, and $A_{min}$ that we test in our simulations. The remaining rows show our calibration of the other variables.\footnote{The code for these simulations is available at \href{https://github.com/OpenSourceEcon/PubDebtNegShocks}{https://github.com/OpenSourceEcon/PubDebtNegShocks}.}

  \begin{table}[htbp]\centering\captionsetup{width=5.6in}
  \caption{\label{TabCalibr}\textbf{Calibration of 2-period-lived agent OG model with promised transfer $\bar{H}$}}
      \begin{threeparttable}
      \begin{tabular}{>{\small}c >{\small}l >{\small}c}
          \hline\hline
          Parameter & \multicolumn{1}{c}{Source to match} & Value(s) \\
          \hline
          $\bar{H}$ & Promised transfer amount & $[0.00, 0.05, 0.11, 0.17]$ \\
          $k_{2,0}$ & Initial period wealth of old household & $[0.11, 0.14, 0.17]$ \\
          $A_{min}$ & Minimum value in support of $A_t$ & $[0.0, 0.75]$ \\
          \hline
          $z_0$ & Initial value of $z_t$ TFP component & $\mu$ \\
          $n_1$ & Exogenous labor supply when young & 1.0 \\
          $n_2$ & Exogenous labor supply when old & 0.0 \\
          $\beta$  & Annual discount factor of 0.96 & 0.29 \\
          $\gamma$ & Coefficient of relative risk aversion between &  2.0 \\
                   & \quad 1.5 and 4.0 &  \\
          $\alpha$ & Capital share of income &  0.35 \\
          $\delta$ & Annual capital depreciation of 0.05 & 0.79 \\
          $\rho$   & AR(1) persistence of normally distributed &  0.21 \\
                   & \quad shock to match annual persistence of 0.95 &       \\
          $\mu$    & AR(1) long-run average $z_t$ level &  0.0 \\
          $\sigma$ & standard deviation of normally distributed $z_t$ &  1.55 \\
                   & \quad to match annual standard deviation of U.S. &  \\
                   & \quad real GDP of 0.49 & \\
          $B_t$    & Exogenous supply of riskless bonds in every & 0 \\
                   & \quad period & \\
          \hline
          $yrs$ & Number of years in a model period & 30 \\
          $T$ & Maximum number of periods to simulate in a & 100 \\
              & \quad given simulation & \\
          $S$ & Number of simulated time series for a given & 3,000 \\
              & \quad parameterization & \\
          \hline\hline
      \end{tabular}
      \begin{tablenotes}
          \scriptsize{\item[]The Technical Appendix \ref{SecTAppCalib} gives a detailed description of the calibration of all parameters.}
      \end{tablenotes}
      \end{threeparttable}
  \end{table}

  In our simulations we study a first layer of the parameter space by simulating combinations of different values of government transfer $\bar{H}\in[0.00, 0.05, 0.11, 0.17]$ and different values of initial wealth $k_{2,0}\in[0.11, 0.14, 0.17]$. We first study the behavior of the economy with these combinations of $\bar{H}$ and $k_{2,0}$ when the total factor productivity value $A_t\equiv e^{z_t}$ is distributed lognormally as described in \eqref{EqModelFirmZAR1} with the range of $A_t$ being $(0,\infty)$. This range of TFP shocks includes very small values close to zero, albeit with low probability, that can create fiscal insolvency when $\bar{H}>0$.

  We then study those same combinations of $\bar{H}$ and $k_{2,0}$, but we assume a truncated support of the total factor productivity process $A_t\in[A_{min},\infty)$, where $A_{min}>0$. We implement this truncated TFP process by assuming that the shocks to the $z_t$ process are truncated normal $TrN()$ with mean 0, standard deviation $\sigma$, and lower bound cutoff $\ve_{t,min}$.\footnote{The truncated normal distribution $TrN(0,\sigma, \ve_{t,min})$ is the non-truncated normal distribution with mean 0 and standard deviation $\sigma$ that is then truncated at $\ve_{t,min}$ and rescaled to sum to 1. Technical Appendix \ref{SecTAppTrNdist} has a description of this distribution.}
  \begin{equation}\label{EqSimsZAR1_trunc}
    \begin{split}
      z_t &= \rho z_{t-1} + (1-\rho)\mu + \ve_t \\
      &\text{where}\quad \rho\in[0,1),\quad\mu\geq 0, \quad\text{and}\quad \ve_t \sim TrN(0,\sigma, \ve_{t,min}) \\
      &\text{and}\quad \ve_{t,min} = \ln(A_{min}) - \rho z_{t-1} - (1 - \rho)\mu
    \end{split}
  \end{equation}

  We use these two alternative scenarios of $A_{min}=0$ versus $A_{min}=0.75$ to study how the properties of the economy change when there are more negative states of the world $A_{min}=0$ versus fewer negative states of the world $A_{min}=0.75$. The economy described by \citet{Blanchard:2019} is a case in which many negative states of the economy are assumed away. Many of the conclusions of \citet{Blanchard:2019} depend critically on these assumptions of relative safety. We show below that the properties of the economy and the costs of government promises change dramatically when the economy is faced with the possibility of more formidable negative shocks.


  \begin{table}[htbp]\centering\captionsetup{width=5.3in}
  \caption{\label{TabInitVal_A0}\textbf{Initial values relative to median values: $H_t = \min\bigl(w_t n_1, \bar{H}\bigr)$, $A_{min}=0.00$ and $z_0=0.0$}}
    \begin{threeparttable}
    \begin{tabular}{>{\small}c| >{\small}c >{\small}c| >{\small}c >{\small}c| >{\small}c >{\small}c}
      \hline\hline
      & \multicolumn{2}{c}{$k_{2,0}=0.11$} & \multicolumn{2}{c}{$k_{2,0}=0.14$} & \multicolumn{2}{c}{$k_{2,0}=0.17$} \\ \cline{2-7}
      & $w_{med}$ & $k_{med}$ & $w_{med}$ & $k_{med}$ & $w_{med}$ & $k_{med}$ \\
      & $\bar{H}/w_{med}$ & $k_{2,0}/k_{med}$ & $\bar{H}/w_{med}$ & $k_{2,0}/k_{med}$ & $\bar{H}/w_{med}$ & $k_{2,0}/k_{med}$ \\
      \hline
      \multirow{2}{*}{$\bar{H}=0.00$}
      & 0.281 & 0.101 & 0.282 & 0.101 & 0.282 & 0.101 \\
      & 0.000 & 1.093 & 0.000 & 1.390 & 0.000 & 1.687 \\
      \hline
      \multirow{2}{*}{$\bar{H}=0.05$}
      & 0.445 & 0.083 & 0.449 & 0.085 & 0.450 & 0.085 \\
      & 0.112 & 1.321 & 0.111 & 1.654 & 0.111 & 2.003 \\
      \hline
      \multirow{2}{*}{$\bar{H}=0.11$}
      & 0.557 & 0.064 & 0.564 & 0.066 & 0.572 & 0.068 \\
      & 0.197 & 1.710 & 0.195 & 2.108 & 0.192 & 2.515 \\
      \hline
      \multirow{2}{*}{$\bar{H}=0.17$}
      & 0.648 & 0.048 & 0.658 & 0.051 & 0.667 & 0.052 \\
      & 0.262 & 2.274 & 0.259 & 2.757 & 0.255 & 3.243 \\
      \hline\hline
    \end{tabular}
    \begin{tablenotes}
      \scriptsize{\item[]$w_{med}$ is the median wage and $k_{med}$ is the median capital stock across all 3,000 simulations before economic shut down.}
    \end{tablenotes}
    \end{threeparttable}
  \end{table}

  \begin{table}[htbp]\centering\captionsetup{width=5.3in}
  \caption{\label{TabInitVal_A75}\textbf{Initial values relative to median values: $H_t = \min\bigl(w_t n_1, \bar{H}\bigr)$, $A_{min}=0.75$ and $z_0=0.0$}}
    \begin{threeparttable}
    \begin{tabular}{>{\small}c| >{\small}c >{\small}c| >{\small}c >{\small}c| >{\small}c >{\small}c}
      \hline\hline
      & \multicolumn{2}{c}{$k_{2,0}=0.11$} & \multicolumn{2}{c}{$k_{2,0}=0.14$} & \multicolumn{2}{c}{$k_{2,0}=0.17$} \\ \cline{2-7}
      & $w_{med}$ & $k_{med}$ & $w_{med}$ & $k_{med}$ & $w_{med}$ & $k_{med}$ \\
      & $\bar{H}/w_{med}$ & $k_{2,0}/k_{med}$ & $\bar{H}/w_{med}$ & $k_{2,0}/k_{med}$ & $\bar{H}/w_{med}$ & $k_{2,0}/k_{med}$ \\
      \hline
      \multirow{2}{*}{$\bar{H}=0.00$}
      & 1.319 & 0.388 & 1.320 & 0.388 & 1.321 & 0.388 \\
      & 0.000 & 0.284 & 0.000 & 0.361 & 0.000 & 0.438 \\
      \hline
      \multirow{2}{*}{$\bar{H}=0.05$}
      & 1.158 & 0.281 & 1.160 & 0.281 & 1.161 & 0.281 \\
      & 0.043 & 0.392 & 0.043 & 0.498 & 0.043 & 0.604 \\
      \hline
      \multirow{2}{*}{$\bar{H}=0.11$}
      & 0.984 & 0.176 & 0.988 & 0.177 & 0.994 & 0.179 \\
      & 0.112 & 0.625 & 0.111 & 0.789 & 0.111 & 0.950 \\
      \hline
      \multirow{2}{*}{$\bar{H}=0.17$}
      & 0.953 & 0.128 & 0.950 & 0.128 & 0.960 & 0.130 \\
      & 0.178 & 0.857 & 0.179 & 1.097 & 0.177 & 1.303 \\
      \hline\hline
    \end{tabular}
    \begin{tablenotes}
      \scriptsize{\item[]$w_{med}$ is the median wage and $k_{med}$ is the median capital stock across all 3,000 simulations before economic shut down.}
    \end{tablenotes}
    \end{threeparttable}
  \end{table}

  \begin{table}[htbp]\centering\captionsetup{width=6.0in}
  \caption{\label{TabInitVal_tA0}\textbf{Initial values relative to median values: $H_t = \tau w_t n_1$, $A_{min}=0.00$ and $z_0=0.0$}}
    \begin{threeparttable}
    \begin{tabular}{>{\small}c| >{\small}c >{\small}c| >{\small}c >{\small}c| >{\small}c >{\small}c}
      \hline\hline
      & \multicolumn{2}{c}{$k_{2,0}=0.11$} & \multicolumn{2}{c}{$k_{2,0}=0.14$} & \multicolumn{2}{c}{$k_{2,0}=0.17$} \\ \cline{2-7}
      & $w_{med}$ & $k_{med}$ & $w_{med}$ & $k_{med}$ & $w_{med}$ & $k_{med}$ \\
      & $H_{t,med}/w_{med}$ & $k_{2,0}/k_{med}$ & $H_{t,med}/w_{med}$ & $k_{2,0}/k_{med}$ & $H_{t,med}/w_{med}$ & $k_{2,0}/k_{med}$ \\
      \hline
      \multirow{2}{*}{$\tau=0.00$}
      & 0.281 & 0.101 & 0.282 & 0.101 & 0.282 & 0.101 \\
      & 0.000 & 1.093 & 0.000 & 1.390 & 0.000 & 1.687 \\
      \hline
      \multirow{2}{*}{$\tau=0.11$}
      & 0.248 & 0.069 & 0.248 & 0.069 & 0.248 & 0.069 \\
      & 0.110 & 1.592 & 0.110 & 2.024 & 0.110 & 2.457 \\
      \hline
      \multirow{2}{*}{$\tau=0.20$}
      & 0.222 & 0.049 & 0.222 & 0.049 & 0.222 & 0.049 \\
      & 0.200 & 2.228 & 0.200 & 2.834 & 0.200 & 3.438 \\
      \hline
      \multirow{2}{*}{$\tau=0.25$}
      & 0.208 & 0.040 & 0.208 & 0.040 & 0.208 & 0.040 \\
      & 0.250 & 2.722 & 0.250 & 3.461 & 0.250 & 4.200 \\
      \hline\hline
    \end{tabular}
    \begin{tablenotes}
      \scriptsize{\item[]$w_{med}$ is the median wage and $k_{med}$ is the median capital stock across all 3,000 simulations before economic shut down.}
    \end{tablenotes}
    \end{threeparttable}
  \end{table}

  \begin{table}[htbp]\centering\captionsetup{width=6.0in}
  \caption{\label{TabInitVal_tA75}\textbf{Initial values relative to median values: $H_t = \tau w_t n_1$, $A_{min}=0.75$ and $z_0=0.0$}}
    \begin{threeparttable}
    \begin{tabular}{>{\small}c| >{\small}c >{\small}c| >{\small}c >{\small}c| >{\small}c >{\small}c}
      \hline\hline
      & \multicolumn{2}{c}{$k_{2,0}=0.11$} & \multicolumn{2}{c}{$k_{2,0}=0.14$} & \multicolumn{2}{c}{$k_{2,0}=0.17$} \\ \cline{2-7}
      & $w_{med}$ & $k_{med}$ & $w_{med}$ & $k_{med}$ & $w_{med}$ & $k_{med}$ \\
      & $H_{t,med}/w_{med}$ & $k_{2,0}/k_{med}$ & $H_{t,med}/w_{med}$ & $k_{2,0}/k_{med}$ & $H_{t,med}/w_{med}$ & $k_{2,0}/k_{med}$ \\
      \hline
      \multirow{2}{*}{$\tau=0.00$}
      & 1.319 & 0.388 & 1.320 & 0.388 & 1.321 & 0.388 \\
      & 0.000 & 0.284 & 0.000 & 0.361 & 0.000 & 0.438 \\
      \hline
      \multirow{2}{*}{$\tau=0.11$}
      & 1.106 & 0.232 & 1.107 & 0.232 & 1.108 & 0.232 \\
      & 0.110 & 0.474 & 0.110 & 0.603 & 0.110 & 0.732 \\
      \hline
      \multirow{2}{*}{$\tau=0.20$}
      & 0.943 & 0.145 & 0.943 & 0.145 & 0.944 & 0.145 \\
      & 0.200 & 0.758 & 0.200 & 0.964 & 0.200 & 1.170 \\
      \hline
      \multirow{2}{*}{$\tau=0.25$}
      & 0.857 & 0.109 & 0.857 & 0.110 & 0.858 & 0.110 \\
      & 0.250 & 1.005 & 0.250 & 1.278 & 0.250 & 1.551 \\
      \hline\hline
    \end{tabular}
    \begin{tablenotes}
      \scriptsize{\item[]$w_{med}$ is the median wage and $k_{med}$ is the median capital stock across all 3,000 simulations before economic shut down.}
    \end{tablenotes}
    \end{threeparttable}
  \end{table}

  \clearpage

  Partial equilibrium effects on measures of central tendency. We test the effect on the measures of central tendency in Tables \ref{TabInitVal_A0}to \ref{TabInitVal_tA75}. In the functions below the varialbe $pol\in\{H_t=\min(w_1 n_1,\bar{H}), \tau\}$.
  \begin{gather*}
    w_{med}\bigl(pol, A_{min}, \bar{H} \:\text{or}\: \tau, k_{2,0}\bigr) \\
    k_{med}\bigl(pol, A_{min}, \bar{H} \:\text{or}\: \tau, k_{2,0}\bigr) \\
    \bar{H}/w_{med}\bigl(pol, A_{min}, \bar{H} \:\text{or}\: \tau, k_{2,0}\bigr) \\
    \text{or} \\
    H_{t,med}/w_{med}\bigl(pol, A_{min}, \bar{H} \:\text{or}\: \tau, k_{2,0}\bigr) \\
    k_{2,0}/k_{med}\bigl(pol, A_{min}, \bar{H} \:\text{or}\: \tau, k_{2,0}\bigr)
  \end{gather*}

  \begin{equation*}
    \begin{split}
      &\frac{\partial w_{med}(A_{min}=0.00)}{\partial\bar{H}}>0, \quad \frac{\partial w_{med}(A_{min}=0.75)}{\partial\bar{H}}<0, \quad \frac{\partial w_{med}}{\partial\tau}<0, \quad \frac{\partial w_{med}}{\partial k_{2,0}}\approx 0, \\
      &\quad \frac{\partial w_{med}}{\partial pol}\leq0, \quad \frac{\partial w_{med}}{\partial A_{min}}>0
    \end{split}
  \end{equation*}

  \begin{equation*}
    \frac{\partial k_{med}}{\partial\bar{H}},\:\frac{\partial k_{med}}{\partial\tau}<0, \quad \frac{\partial k_{med}}{\partial k_{2,0}}\approx 0, \quad \frac{\partial k_{med}}{\partial pol}\leq0, \quad \frac{\partial k_{med}}{\partial A_{min}}>0
  \end{equation*}

  \begin{equation*}
    \begin{split}
      &\frac{\partial\bar{H}/w_{med}}{\partial\bar{H}},\:\frac{\partial H_{t,med}/w_{med}}{\partial\tau}>0, \quad \frac{\partial\bar{H}/w_{med}}{\partial k_{2,0}},\:\frac{\partial H_{t,med}/w_{med}}{\partial k_{2,0}}\approx 0, \\
      &\quad \frac{\partial H_{t,med}/w_{med}(A_{min}=0.00)}{\partial pol}\approx 0, \quad \frac{\partial H_{t,med}/w_{med}(A_{min}=0.75)}{\partial pol}\geq 0, \\
      &\quad \frac{\partial H_{t,med}/w_{med}(pol=\bar{H})}{\partial A_{min}}\leq 0, \quad \frac{\partial H_{t,med}/w_{med}(pol=\tau)}{\partial A_{min}}= 0
    \end{split}
  \end{equation*}

  \begin{equation*}
    \begin{split}
      &\frac{\partial k_{2,0}/k_{med}}{\partial\bar{H}},\:\frac{\partial k_{2,0}/k_{med}}{\partial\tau}>0, \quad \frac{\partial k_{2,0}/k_{med}}{\partial k_{2,0}}> 0 \\
      &\quad \frac{\partial k_{2,0}/k_{med}}{\partial pol}> 0, \quad \frac{\partial k_{2,0}/k_{med}}{\partial A_{min}}< 0
    \end{split}
  \end{equation*}

  \newpage

  \begin{table}[htbp]\centering\captionsetup{width=4.6in}
  \caption{\label{TabPer2GO_A0}\textbf{Periods to shut down simulation statistics: $A_{min}=0.00$ and $z_0=0.0$}}
    \begin{threeparttable}
    \begin{tabular}{>{\small}c >{\small}l| >{\small}c >{\small}c| >{\small}c >{\small}c| >{\small}c >{\small}c}
      \hline\hline
      & & \multicolumn{2}{c}{$k_{2,0}=0.11$} & \multicolumn{2}{c}{$k_{2,0}=0.14$} & \multicolumn{2}{c}{$k_{2,0}=0.17$} \\ \cline{3-8}
      & & Periods & CDF & Periods & CDF & Periods & CDF \\
      \hline
      \multirow{4}{*}{$\bar{H}=0.00$}
      & min & 100 & 1.000 & 100 & 1.000 & 100 & 1.000 \\
      & med & 100 & 1.000 & 100 & 1.000 & 100 & 1.000 \\
      & mean & 100 & 1.000 & 100 & 1.000 & 100 & 1.000 \\
      & max & 100 & 1.000 & 100 & 1.000 & 100 & 1.000 \\
      \hline
      \multirow{4}{*}{$\bar{H}=0.05$}
      & min & 1 & 0.160 & 1 & 0.152 & 1 & 0.148 \\
      & med & 4 & 0.517 & 4 & 0.512 & 4 & 0.507 \\
      & mean & 6.1 & 0.693 & 6.2 & 0.689 & 6.2 & 0.686 \\
      & max & 50 & 1.000 & 50 & 1.000 & 50 & 1.000 \\
      \hline
      \multirow{4}{*}{$\bar{H}=0.11$}
      & min & 1 & 0.344 & 1 & 0.328 & 1 & 0.317 \\
      & med & 2 & 0.534 & 2 & 0.522 & 2 & 0.512 \\
      & mean & 3.5 & 0.713 & 3.6 & 0.705 & 3.6 & 0.697 \\
      & max & 27 & 1.000 & 27 & 1.000 & 27 & 1.000 \\
      \hline
      \multirow{4}{*}{$\bar{H}=0.17$}
      & min & 1 & 0.498 & 1 & 0.474 & 1 & 0.459 \\
      & med & 2 & 0.683 & 2 & 0.670 & 2 & 0.658 \\
      & mean & 2.5 & 0.758 & 2.6 & 0.745 & 2.6 & 0.732 \\
      & max & 21 & 1.000 & 21 & 1.000 & 21 & 1.000 \\
      \hline\hline
    \end{tabular}
    \begin{tablenotes}
      \scriptsize{\item[]The ``min", ``med", ``mean", and ``max" rows in the ``Periods" column represent the minimum, median, mean, and maximum number of periods, respectively, in which the simulated time series hit the economic shut down. The ``CDF" column represents the percent of simulations that shut down in $t$ periods or less, where $t$ is the value in the ``Periods" column. For the CDF value of the ``mean" row, we used linear interpolation.}
    \end{tablenotes}
    \end{threeparttable}
  \end{table}

  \begin{table}[htbp]\centering\captionsetup{width=4.6in}
  \caption{\label{TabPer2GO_A75}\textbf{Periods to shut down simulation statistics: $A_{min}=0.75$ and $z_0=0.0$}}
    \begin{threeparttable}
    \begin{tabular}{>{\small}c >{\small}l| >{\small}c >{\small}c| >{\small}c >{\small}c| >{\small}c >{\small}c}
      \hline\hline
      & & \multicolumn{2}{c}{$k_{2,0}=0.11$} & \multicolumn{2}{c}{$k_{2,0}=0.14$} & \multicolumn{2}{c}{$k_{2,0}=0.17$} \\ \cline{3-8}
      & & Periods & CDF & Periods & CDF & Periods & CDF \\
      \hline
      \multirow{4}{*}{$\bar{H}=0.00$}
      & min & 100 & 1.000 & 100 & 1.000 & 100 & 1.000 \\
      & med & 100 & 1.000 & 100 & 1.000 & 100 & 1.000 \\
      & mean & 100 & 1.000 & 100 & 1.000 & 100 & 1.000 \\
      & max & 100 & 1.000 & 100 & 1.000 & 100 & 1.000 \\
      \hline
      \multirow{4}{*}{$\bar{H}=0.05$}
      & min & 5 & 0.000 & 5 & 0.000 & 5 & 0.000 \\
      & med & 100 & 1.000 & 100 & 1.000 & 100 & 1.000 \\
      & mean & 99.8 & 0.008 & 99.8 & 0.007 & 99.8 & 0.007 \\
      & max & 100 & 1.000 & 100 & 1.000 & 100 & 1.000 \\
      \hline
      \multirow{4}{*}{$\bar{H}=0.11$}
      & min & 2 & 0.108 & 2 & 0.096 & 2 & 0.086 \\
      & med & 23 & 0.500 & 25 & 0.506 & 25 & 0.502 \\
      & mean & 34.1 & 0.620 & 34.9 & 0.608 & 35.2 & 0.613 \\
      & max & 100 & 1.000 & 100 & 1.000 & 100 & 1.000 \\
      \hline
      \multirow{4}{*}{$\bar{H}=0.17$}
      & min & 1 & 0.302 & 1 & 0.261 & 1 & 0.228 \\
      & med & 3 & 0.506 & 4 & 0.521 & 4 & 0.501 \\
      & mean & 8.5 & 0.692 & 9.0 & 0.674 & 9.4 & 0.680 \\
      & max & 100 & 1.000 & 100 & 1.000 & 100 & 1.000 \\
      \hline\hline
    \end{tabular}
    \begin{tablenotes}
      \scriptsize{\item[]The ``min", ``med", ``mean", and ``max" rows in the ``Periods" column represent the minimum, median, mean, and maximum number of periods, respectively, in which the simulated time series hit the economic shut down. The ``CDF" column represents the percent of simulations that shut down in $t$ periods or less, where $t$ is the value in the ``Periods" column. For the CDF value of the ``mean" row, we used linear interpolation.}
    \end{tablenotes}
    \end{threeparttable}
  \end{table}

  \clearpage

  Partial equilibrium effects on statistics on periods to shut down across simulations. We test the effect on the statistics on periods to shut down in Tables \ref{TabPer2GO_A0} and \ref{TabPer2GO_A75}.

  \begin{equation*}
    \begin{split}
      &\frac{\partial\text{min}}{\partial\bar{H}}< 0, \quad \frac{\partial\text{min}}{\partial k_{2,0}}=0, \frac{\partial\text{min}}{\partial A_{min}}> 0 \\
      &\frac{\partial\text{med}}{\partial\bar{H}}> 0, \quad \frac{\partial\text{med}}{\partial k_{2,0}}> 0, \frac{\partial\text{med}}{\partial A_{min}}> 0 \\
      &\frac{\partial\text{mean}}{\partial\bar{H}}> 0, \quad \frac{\partial\text{mean}}{\partial k_{2,0}}> 0, \frac{\partial\text{mean}}{\partial A_{min}}> 0 \\
      &\frac{\partial\text{max}}{\partial\bar{H}}> 0, \quad \frac{\partial\text{max}}{\partial k_{2,0}}> 0, \frac{\partial\text{max}}{\partial A_{min}}> 0
    \end{split}
  \end{equation*}

  \newpage

  \begin{table}[htbp]\centering\captionsetup{width=4.6in}
  \caption{\label{TabRiskl_A0}\textbf{Annualized Riskless return $\bar{r}_{t,an}$ simulation statistics: $H_t=\min\bigl(w_t n_1, \bar{H}\bigr)$, $A_{min}=0.00$ and $z_0=0.0$}}
    \begin{threeparttable}
    \begin{tabular}{>{\small}c >{\small}l| >{\small}c >{\small}c| >{\small}c >{\small}c| >{\small}c >{\small}c}
      \hline\hline
      & & \multicolumn{2}{c}{$k_{2,0}=0.11$} & \multicolumn{2}{c}{$k_{2,0}=0.14$} & \multicolumn{2}{c}{$k_{2,0}=0.17$} \\ \cline{3-8}
      & & $\bar{r}_{t,an}$ & CDF & $\bar{r}_{t,an}$ & CDF & $\bar{r}_{t,an}$ & CDF \\
      \hline
      \multirow{5}{*}{$\bar{H}=0.00$}
      & $t=0$ & -2.06\% & 0.483 & -2.12\% & 0.459 & -2.18\% & 0.440 \\
      & min & -4.64\% & 0.000 & -4.64\% & 0.000 & -4.64\% & 0.000 \\
      & med & -2.01\% & 0.500 & -2.01\% & 0.500 & -2.01\% & 0.500 \\
      & mean & -1.58\% & 0.645 & -1.58\% & 0.645 & -1.59\% & 0.645 \\
      & max & 19.01\% & 1.000 & 19.01\% & 1.000 & 19.01\% & 1.000 \\
      \hline
      \multirow{5}{*}{$\bar{H}=0.05$}
      & $t=0$ & -1.47\% & 0.589 & -1.54\% & 0.563 & -1.60\% & 0.541 \\
      & min & -4.39\% & 0.000 & -4.39\% & 0.000 & -4.39\% & 0.000 \\
      & med & -1.69\% & 0.500 & -1.70\% & 0.500 & -1.71\% & 0.500 \\
      & mean & -1.14\% & 0.715 & -1.14\% & 0.720 & -1.14\% & 0.720 \\
      & max & 36.56\% & 1.000 & 36.52\% & 1.000 & 36.49\% & 1.000 \\
      \hline
      \multirow{5}{*}{$\bar{H}=0.11$}
      & $t=0$ & -1.71\% & 0.651 & -1.80\% & 0.609 & -1.87\% & 0.580 \\
      & min & -4.38\% & 0.000 & -4.38\% & 0.000 & -4.38\% & 0.000 \\
      & med & -1.99\% & 0.500 & -2.00\% & 0.500 & -2.01\% & 0.500 \\
      & mean & -1.42\% & 0.750 & -1.40\% & 0.759 & -1.43\% & 0.756 \\
      & max & 34.67\% & 1.000 & 36.29\% & 1.000 & 32.10\% & 1.000 \\
      \hline
      \multirow{5}{*}{$\bar{H}=0.17$}
      & $t=0$ & -1.53\% & 0.743 & -1.71\% & 0.704 & -1.83\% & 0.674 \\
      & min & -4.37\% & 0.000 & -4.37\% & 0.000 & -4.37\% & 0.000 \\
      & med & -2.13\% & 0.500 & -2.16\% & 0.500 & -2.18\% & 0.500 \\
      & mean & -1.51\% & 0.751 & -1.53\% & 0.754 & -1.59\% & 0.749 \\
      & max & 35.69\% & 1.000 & 41.27\% & 1.000 & 35.76\% & 1.000 \\
      \hline\hline
    \end{tabular}
    \begin{tablenotes}
      \scriptsize{\item[]All riskless returns $\bar{r}_{t,an}$ are reported in percentage rates. The rate of return 0.0206 is reported in this table as 2.06\%.}
    \end{tablenotes}
    \end{threeparttable}
  \end{table}

  \begin{table}[htbp]\centering\captionsetup{width=4.6in}
  \caption{\label{TabRiskl_A75}\textbf{Annualized Riskless return $\bar{r}_{t,an}$ simulation statistics: $H_t=\min\bigl(w_t n_1, \bar{H}\bigr)$, $A_{min}=0.75$ and $z_0=0.0$}}
    \begin{threeparttable}
    \begin{tabular}{>{\small}c >{\small}l| >{\small}c >{\small}c| >{\small}c >{\small}c| >{\small}c >{\small}c}
      \hline\hline
      & & \multicolumn{2}{c}{$k_{2,0}=0.11$} & \multicolumn{2}{c}{$k_{2,0}=0.14$} & \multicolumn{2}{c}{$k_{2,0}=0.17$} \\ \cline{3-8}
      & & $\bar{r}_{t,an}$ & CDF & $\bar{r}_{t,an}$ & CDF & $\bar{r}_{t,an}$ & CDF \\
      \hline
      \multirow{5}{*}{$\bar{H}=0.00$}
      & $t=0$ & 3.91\% & 0.955 & 3.69\% & 0.940 & 3.51\% & 0.926 \\
      & min & -4.60\% & 0.000 & -4.60\% & 0.000 & -4.60\% & 0.000 \\
      & med & 0.47\% & 0.500 & 0.47\% & 0.500 & 0.47\% & 0.500 \\
      & mean & 0.52\% & 0.509 & 0.52\% & 0.509 & 0.52\% & 0.509 \\
      & max & 6.19\% & 1.000 & 6.19\% & 1.000 & 6.19\% & 1.000 \\
      \hline
      \multirow{5}{*}{$\bar{H}=0.05$}
      & $t=0$ & 5.40\% & 0.912 & 5.06\% & 0.895 & 4.80\% & 0.879 \\
      & min & -4.60\% & 0.000 & -4.60\% & 0.000 & -4.60\% & 0.000 \\
      & med & 1.15\% & 0.500 & 1.14\% & 0.500 & 1.14\% & 0.500 \\
      & mean & 1.43\% & 0.537 & 1.43\% & 0.537 & 1.42\% & 0.537 \\
      & max & 41.95\% & 1.000 & 41.95\% & 1.000 & 41.95\% & 1.000 \\
      \hline
      \multirow{5}{*}{$\bar{H}=0.11$}
      & $t=0$ & 7.62\% & 0.855 & 7.07\% & 0.836 & 6.65\% & 0.819 \\
      & min & -4.57\% & 0.000 & -4.57\% & 0.000 & -4.57\% & 0.000 \\
      & med & 2.26\% & 0.500 & 2.24\% & 0.500 & 2.23\% & 0.500 \\
      & mean & 3.20\% & 0.588 & 3.16\% & 0.587 & 3.14\% & 0.586 \\
      & max & 70.01\% & 1.000 & 76.72\% & 1.000 & 74.84\% & 1.000 \\
      \hline
      \multirow{5}{*}{$\bar{H}=0.17$}
      & $t=0$ & 9.56\% & 0.859 & 8.92\% & 0.838 & 8.43\% & 0.822 \\
      & min & -4.29\% & 0.000 & -4.29\% & 0.000 & -4.30\% & 0.000 \\
      & med & 3.13\% & 0.500 & 3.13\% & 0.500 & 3.09\% & 0.500 \\
      & mean & 4.30\% & 0.584 & 4.26\% & 0.582 & 4.21\% & 0.584 \\
      & max & 69.65\% & 1.000 & 67.07\% & 1.000 & 70.83\% & 1.000 \\
      \hline\hline
    \end{tabular}
    \begin{tablenotes}
      \scriptsize{\item[]All riskless returns $\bar{r}_{t,an}$ are reported in percentage rates. The rate of return 0.0206 is reported in this table as 2.06\%.}
    \end{tablenotes}
    \end{threeparttable}
  \end{table}

  \begin{table}[htbp]\centering\captionsetup{width=4.6in}
  \caption{\label{TabRiskl_tA0}\textbf{Annualized Riskless return $\bar{r}_{t,an}$ simulation statistics: $H_t = \tau w_t n_1$, $A_{min}=0.00$ and $z_0=0.0$}}
    \begin{threeparttable}
    \begin{tabular}{>{\small}c >{\small}l| >{\small}c >{\small}c| >{\small}c >{\small}c| >{\small}c >{\small}c}
      \hline\hline
      & & \multicolumn{2}{c}{$k_{2,0}=0.11$} & \multicolumn{2}{c}{$k_{2,0}=0.14$} & \multicolumn{2}{c}{$k_{2,0}=0.17$} \\ \cline{3-8}
      & & $\bar{r}_{t,an}$ & CDF & $\bar{r}_{t,an}$ & CDF & $\bar{r}_{t,an}$ & CDF \\
      \hline
      \multirow{5}{*}{$\tau=0.00$}
      & $t=0$ & -2.06\% & 0.483 & -2.12\% & 0.459 & -2.18\% & 0.440 \\
      & min & -4.64\% & 0.000 & -4.64\% & 0.000 & -4.64\% & 0.000 \\
      & med & -2.01\% & 0.500 & -2.01\% & 0.500 & -2.01\% & 0.500 \\
      & mean & -1.58\% & 0.645 & -1.58\% & 0.645 & -1.59\% & 0.645 \\
      & max & 20.00\% & 1.000 & 20.00\% & 1.000 & 20.00\% & 1.000 \\
      \hline
      \multirow{5}{*}{$\tau=0.11$}
      & $t=0$ & -2.08\% & 0.446 & -2.15\% & 0.423 & -2.20\% & 0.405 \\
      & min & -4.63\% & 0.000 & -4.63\% & 0.000 & -4.63\% & 0.000 \\
      & med & -1.93\% & 0.500 & -1.93\% & 0.500 & -1.93\% & 0.500 \\
      & mean & -1.34\% & 0.678 & -1.34\% & 0.678 & -1.35\% & 0.678 \\
      & max & 22.45\% & 1.000 & 22.45\% & 1.000 & 22.45\% & 1.000 \\
      \hline
      \multirow{5}{*}{$\tau=0.20$}
      & $t=0$ & -2.07\% & 0.415 & -2.13\% & 0.393 & -2.19\% & 0.374 \\
      & min & -4.61\% & 0.000 & -4.61\% & 0.000 & -4.61\% & 0.000 \\
      & med & -1.81\% & 0.500 & -1.81\% & 0.500 & -1.81\% & 0.500 \\
      & mean & -1.06\% & 0.705 & -1.06\% & 0.705 & -1.06\% & 0.705 \\
      & max & 24.51\% & 1.000 & 24.51\% & 1.000 & 24.51\% & 1.000 \\
      \hline
      \multirow{5}{*}{$\tau=0.25$}
      & $t=0$ & -2.04\% & 0.397 & -2.11\% & 0.374 & -2.17\% & 0.356 \\
      & min & -4.61\% & 0.000 & -4.61\% & 0.000 & -4.61\% & 0.000 \\
      & med & -1.72\% & 0.500 & -1.72\% & 0.500 & -1.72\% & 0.500 \\
      & mean & -0.86\% & 0.721 & -0.86\% & 0.721 & -0.86\% & 0.721 \\
      & max & 25.71\% & 1.000 & 25.71\% & 1.000 & 25.71\% & 1.000 \\
      \hline\hline
    \end{tabular}
    \begin{tablenotes}
      \scriptsize{\item[]All riskless returns $\bar{r}_{t,an}$ are reported in percentage rates. The rate of return 0.0206 is reported in this table as 2.06\%.}
    \end{tablenotes}
    \end{threeparttable}
  \end{table}

  \begin{table}[htbp]\centering\captionsetup{width=4.6in}
  \caption{\label{TabRiskl_tA75}\textbf{Annualized Riskless return $\bar{r}_{t,an}$ simulation statistics: $H_t = \tau w_t n_1$, $A_{min}=0.75$ and $z_0=0.0$}}
    \begin{threeparttable}
    \begin{tabular}{>{\small}c >{\small}l| >{\small}c >{\small}c| >{\small}c >{\small}c| >{\small}c >{\small}c}
      \hline\hline
      & & \multicolumn{2}{c}{$k_{2,0}=0.11$} & \multicolumn{2}{c}{$k_{2,0}=0.14$} & \multicolumn{2}{c}{$k_{2,0}=0.17$} \\ \cline{3-8}
      & & $\bar{r}_{t,an}$ & CDF & $\bar{r}_{t,an}$ & CDF & $\bar{r}_{t,an}$ & CDF \\
      \hline
      \multirow{5}{*}{$\tau=0.00$}
      & $t=0$ & 3.91\% & 0.955 & 3.69\% & 0.940 & 3.51\% & 0.926 \\
      & min & -4.60\% & 0.000 & -4.60\% & 0.000 & -4.60\% & 0.000 \\
      & med & 0.47\% & 0.500 & 0.47\% & 0.500 & 0.47\% & 0.500 \\
      & mean & 0.52\% & 0.509 & 0.52\% & 0.509 & 0.52\% & 0.509 \\
      & max & 6.19\% & 1.000 & 6.19\% & 1.000 & 6.19\% & 1.000 \\
      \hline
      \multirow{5}{*}{$\tau=0.11$}
      & $t=0$ & 4.62\% & 0.916 & 4.38\% & 0.896 & 4.19\% & 0.878 \\
      & min & -4.59\% & 0.000 & -4.59\% & 0.000 & -4.59\% & 0.000 \\
      & med & 1.31\% & 0.500 & 1.31\% & 0.500 & 1.31\% & 0.500 \\
      & mean & 1.33\% & 0.503 & 1.33\% & 0.503 & 1.33\% & 0.503 \\
      & max & 7.80\% & 1.000 & 7.80\% & 1.000 & 7.80\% & 1.000 \\
      \hline
      \multirow{5}{*}{$\tau=0.20$}
      & $t=0$ & 5.23\% & 0.872 & 4.98\% & 0.849 & 4.79\% & 0.829 \\
      & min & -4.57\% & 0.000 & -4.57\% & 0.000 & -4.57\% & 0.000 \\
      & med & 2.15\% & 0.500 & 2.15\% & 0.500 & 2.15\% & 0.500 \\
      & mean & 2.13\% & 0.497 & 2.12\% & 0.497 & 2.12\% & 0.497 \\
      & max & 9.20\% & 1.000 & 9.20\% & 1.000 & 9.20\% & 1.000 \\
      \hline
      \multirow{5}{*}{$\tau=0.25$}
      & $t=0$ & 5.59\% & 0.844 & 5.34\% & 0.819 & 5.13\% & 0.798 \\
      & min & -4.55\% & 0.000 & -4.55\% & 0.000 & -4.55\% & 0.000 \\
      & med & 2.68\% & 0.500 & 2.68\% & 0.500 & 2.68\% & 0.500 \\
      & mean & 2.63\% & 0.493 & 2.63\% & 0.493 & 2.63\% & 0.493 \\
      & max & 10.03\% & 1.000 & 10.03\% & 1.000 & 10.03\% & 1.000 \\
      \hline\hline
    \end{tabular}
    \begin{tablenotes}
      \scriptsize{\item[]All riskless returns $\bar{r}_{t,an}$ are reported in percentage rates. The rate of return 0.0206 is reported in this table as 2.06\%.}
    \end{tablenotes}
    \end{threeparttable}
  \end{table}

  \begin{table}[htbp]\centering\captionsetup{width=6.0in}
    \caption{\label{TabEqPrem_A0}\textbf{Components of the equity premium in annual terms: $H_t = \min\bigl(w_t n_1, \bar{H}\bigr)$, $A_{min}=0.00$ and $z_0=0.0$}}
    \begin{threeparttable}
    \begin{tabular}{>{\small}l| >{\small}l| >{\small}c| >{\small}c| >{\small}c}
      \hline\hline
      & & $k_{2,0}=0.11$ & $k_{2,0}=0.14$ & $k_{2,0}=0.17$ \\
      \hline
      \multirow{5}{*}{$\bar{H}=0.00$} & \quad Avg. $E[R_{t+1}]$ & 102.7\% & 102.7\% & 102.7\% \\
      & \quad $\sigma(R_{t+1})$ & 5.090 & 5.090 & 5.089 \\
      & \quad Avg. $\bar{R_t}$ & 98.4\% & 98.4\% & 98.4\% \\
      & \quad Avg. eq. prem. $E[R_{t+1}] - \bar{R_t]}$ & 4.3\% & 4.3\% & 4.3\% \\
      & \quad Avg. Sharpe ratio $\frac{E[R_{t+1}] - \bar{R_t]}}{\sigma(R_{t+1})}$ & 0.848 & 0.848 & 0.847 \\
      \hline
      \multirow{5}{*}{$\bar{H}=0.05$} & \quad Avg. $E[R_{t+1}]$ & 104.1\% & 104.0\% & 103.9\% \\
      & \quad $\sigma(R_{t+1})$ & 5.546 & 5.538 & 5.525 \\
      & \quad Avg. $\bar{R_t}$ & 98.9\% & 98.9\% & 98.9\% \\
      & \quad Avg. eq. prem. $E[R_{t+1}] - \bar{R_t]}$ & 5.2\% & 5.1\% & 5.1\% \\
      & \quad Avg. Sharpe ratio $\frac{E[R_{t+1}] - \bar{R_t]}}{\sigma(R_{t+1})}$ & 0.938 & 0.925 & 0.918 \\
      \hline
      \multirow{5}{*}{$\bar{H}=0.11$} & \quad Avg. $E[R_{t+1}]$ & 105.1\% & 104.9\% & 104.8\% \\
      & \quad $\sigma(R_{t+1})$ & 5.593 & 5.516 & 5.473 \\
      & \quad Avg. $\bar{R_t}$ & 98.6\% & 98.6\% & 98.6\% \\
      & \quad Avg. eq. prem. $E[R_{t+1}] - \bar{R_t]}$ & 6.6\% & 6.3\% & 6.2\% \\
      & \quad Avg. Sharpe ratio $\frac{E[R_{t+1}] - \bar{R_t]}}{\sigma(R_{t+1})}$ & 1.171 & 1.148 & 1.140 \\
      \hline
      \multirow{5}{*}{$\bar{H}=0.17$} & \quad Avg. $E[R_{t+1}]$ & 106.4\% & 106.1\% & 105.8\% \\
      & \quad $\sigma(R_{t+1})$ & 5.789 & 5.627 & 5.553 \\
      & \quad Avg. $\bar{R_t}$ & 98.5\% & 98.5\% & 98.4\% \\
      & \quad Avg. eq. prem. $E[R_{t+1}] - \bar{R_t]}$ & 7.9\% & 7.6\% & 7.4\% \\
      & \quad Avg. Sharpe ratio $\frac{E[R_{t+1}] - \bar{R_t]}}{\sigma(R_{t+1})}$ & 1.367 & 1.352 & 1.337 \\
      \hline\hline
    \end{tabular}
    \begin{tablenotes}
      \scriptsize{\item[]The gross risky one-period return on capital is $R_{t+1} = 1 + r_{t+1}$ and the average expected gross risky return (Avg. $E[R_{t+1}]$) is the average value of $R_{t+1}$ across simulations. The annualized gross risky one-period return is $(R_{t+1})^{1/30}$. The standard deviation of $R_{t+1}$ is just the standard deviation of its realized value across simulations. The average riskless gross return (Avg. $\bar{R}_t$) is the average value across simulations, where $\bar{R}_t=1+\bar{r}_t$.}
    \end{tablenotes}
    \end{threeparttable}
  \end{table}

  \begin{table}[htbp]\centering\captionsetup{width=6.0in}
    \caption{\label{TabEqPrem_A75}\textbf{Components of the equity premium in annual terms: $H_t = \min\bigl(w_t n_1, \bar{H}\bigr)$, $A_{min}=0.75$ and $z_0=0.0$}}
    \begin{threeparttable}
    \begin{tabular}{>{\small}l| >{\small}l| >{\small}c| >{\small}c| >{\small}c}
      \hline\hline
      & & $k_{2,0}=0.11$ & $k_{2,0}=0.14$ & $k_{2,0}=0.17$ \\
      \hline
      \multirow{5}{*}{$\bar{H}=0.00$} & \quad Avg. $E[R_{t+1}]$ & 102.8\% & 102.8\% & 102.8\% \\
      & \quad $\sigma(R_{t+1})$ & 3.944 & 3.941 & 3.938 \\
      & \quad Avg. $\bar{R_t}$ & 100.5\% & 100.5\% & 100.5\% \\
      & \quad Avg. eq. prem. $E[R_{t+1}] - \bar{R_t]}$ & 2.3\% & 2.3\% & 2.3\% \\
      & \quad Avg. Sharpe ratio $\frac{E[R_{t+1}] - \bar{R_t]}}{\sigma(R_{t+1})}$ & 0.584 & 0.583 & 0.583 \\
      \hline
      \multirow{5}{*}{$\bar{H}=0.05$} & \quad Avg. $E[R_{t+1}]$ & 103.7\% & 103.7\% & 103.6\% \\
      & \quad $\sigma(R_{t+1})$ & 4.418 & 4.412 & 4.408 \\
      & \quad Avg. $\bar{R_t}$ & 101.4\% & 101.4\% & 101.4\% \\
      & \quad Avg. eq. prem. $E[R_{t+1}] - \bar{R_t]}$ & 2.2\% & 2.2\% & 2.2\% \\
      & \quad Avg. Sharpe ratio $\frac{E[R_{t+1}] - \bar{R_t]}}{\sigma(R_{t+1})}$ & 0.504 & 0.504 & 0.504 \\
      \hline
      \multirow{5}{*}{$\bar{H}=0.11$} & \quad Avg. $E[R_{t+1}]$ & 105.1\% & 105.0\% & 105.0\% \\
      & \quad $\sigma(R_{t+1})$ & 5.462 & 5.436 & 5.408 \\
      & \quad Avg. $\bar{R_t}$ & 103.2\% & 103.2\% & 103.1\% \\
      & \quad Avg. eq. prem. $E[R_{t+1}] - \bar{R_t]}$ & 1.9\% & 1.9\% & 1.9\% \\
      & \quad Avg. Sharpe ratio $\frac{E[R_{t+1}] - \bar{R_t]}}{\sigma(R_{t+1})}$ & 0.345 & 0.347 & 0.347 \\
      \hline
      \multirow{5}{*}{$\bar{H}=0.17$} & \quad Avg. $E[R_{t+1}]$ & 106.3\% & 106.2\% & 106.1\% \\
      & \quad $\sigma(R_{t+1})$ & 6.257 & 6.131 & 6.030 \\
      & \quad Avg. $\bar{R_t}$ & 104.3\% & 104.3\% & 104.2\% \\
      & \quad Avg. eq. prem. $E[R_{t+1}] - \bar{R_t]}$ & 2.0\% & 1.9\% & 1.9\% \\
      & \quad Avg. Sharpe ratio $\frac{E[R_{t+1}] - \bar{R_t]}}{\sigma(R_{t+1})}$ & 0.324 & 0.318 & 0.310 \\
      \hline\hline
    \end{tabular}
    \begin{tablenotes}
      \scriptsize{\item[]The gross risky one-period return on capital is $R_{t+1} = 1 + r_{t+1}$ and the average expected gross risky return (Avg. $E[R_{t+1}]$) is the average value of $R_{t+1}$ across simulations. The annualized gross risky one-period return is $(R_{t+1})^{1/30}$. The standard deviation of $R_{t+1}$ is just the standard deviation of its realized value across simulations. The average riskless gross return (Avg. $\bar{R}_t$) is the average value across simulations, where $\bar{R}_t=1+\bar{r}_t$.}
    \end{tablenotes}
    \end{threeparttable}
  \end{table}

  \begin{table}[htbp]\centering\captionsetup{width=6.0in}
    \caption{\label{TabEqPrem_tA0}\textbf{Components of the equity premium in annual terms: $H_t = \tau w_t n_1$, $A_{min}=0.00$ and $z_0=0.0$}}
    \begin{threeparttable}
    \begin{tabular}{>{\small}l| >{\small}l| >{\small}c| >{\small}c| >{\small}c}
      \hline\hline
      & & $k_{2,0}=0.11$ & $k_{2,0}=0.14$ & $k_{2,0}=0.17$ \\
      \hline
      \multirow{5}{*}{$\tau=0.00$} & \quad Avg. $E[R_{t+1}]$ & 102.7\% & 102.7\% & 102.7\% \\
      & \quad $\sigma(R_{t+1})$ & 5.090 & 5.090 & 5.089 \\
      & \quad Avg. $\bar{R_t}$ & 98.4\% & 98.4\% & 98.4\% \\
      & \quad Avg. eq. prem. $E[R_{t+1}] - \bar{R_t]}$ & 4.3\% & 4.3\% & 4.3\% \\
      & \quad Avg. Sharpe ratio $\frac{E[R_{t+1}] - \bar{R_t]}}{\sigma(R_{t+1})}$ & 0.848 & 0.848 & 0.847 \\
      \hline
      \multirow{5}{*}{$\tau=0.11$} & \quad Avg. $E[R_{t+1}]$ & 103.4\% & 103.4\% & 103.4\% \\
      & \quad $\sigma(R_{t+1})$ & 5.273 & 5.272 & 5.272 \\
      & \quad Avg. $\bar{R_t}$ & 98.7\% & 98.7\% & 98.7\% \\
      & \quad Avg. eq. prem. $E[R_{t+1}] - \bar{R_t]}$ & 4.7\% & 4.7\% & 4.7\% \\
      & \quad Avg. Sharpe ratio $\frac{E[R_{t+1}] - \bar{R_t]}}{\sigma(R_{t+1})}$ & 0.895 & 0.895 & 0.894 \\
      \hline
      \multirow{5}{*}{$\tau=0.20$} & \quad Avg. $E[R_{t+1}]$ & 104.0\% & 103.9\% & 103.9\% \\
      & \quad $\sigma(R_{t+1})$ & 5.417 & 5.417 & 5.416 \\
      & \quad Avg. $\bar{R_t}$ & 98.9\% & 98.9\% & 98.9\% \\
      & \quad Avg. eq. prem. $E[R_{t+1}] - \bar{R_t]}$ & 5.0\% & 5.0\% & 5.0\% \\
      & \quad Avg. Sharpe ratio $\frac{E[R_{t+1}] - \bar{R_t]}}{\sigma(R_{t+1})}$ & 0.925 & 0.925 & 0.925 \\
      \hline
      \multirow{5}{*}{$\tau=0.25$} & \quad Avg. $E[R_{t+1}]$ & 104.3\% & 104.3\% & 104.3\% \\
      & \quad $\sigma(R_{t+1})$ & 5.493 & 5.493 & 5.493 \\
      & \quad Avg. $\bar{R_t}$ & 99.1\% & 99.1\% & 99.1\% \\
      & \quad Avg. eq. prem. $E[R_{t+1}] - \bar{R_t]}$ & 5.2\% & 5.2\% & 5.1\% \\
      & \quad Avg. Sharpe ratio $\frac{E[R_{t+1}] - \bar{R_t]}}{\sigma(R_{t+1})}$ & 0.938 & 0.938 & 0.938 \\
      \hline\hline
    \end{tabular}
    \begin{tablenotes}
      \scriptsize{\item[]The gross risky one-period return on capital is $R_{t+1} = 1 + r_{t+1}$ and the average expected gross risky return (Avg. $E[R_{t+1}]$) is the average value of $R_{t+1}$ across simulations. The annualized gross risky one-period return is $(R_{t+1})^{1/30}$. The standard deviation of $R_{t+1}$ is just the standard deviation of its realized value across simulations. The average riskless gross return (Avg. $\bar{R}_t$) is the average value across simulations, where $\bar{R}_t=1+\bar{r}_t$.}
    \end{tablenotes}
    \end{threeparttable}
  \end{table}

  \begin{table}[htbp]\centering\captionsetup{width=6.0in}
    \caption{\label{TabEqPrem_tA75}\textbf{Components of the equity premium in annual terms: $H_t = \tau w_t n_1$, $A_{min}=0.75$ and $z_0=0.0$}}
    \begin{threeparttable}
    \begin{tabular}{>{\small}l| >{\small}l| >{\small}c| >{\small}c| >{\small}c}
      \hline\hline
      & & $k_{2,0}=0.11$ & $k_{2,0}=0.14$ & $k_{2,0}=0.17$ \\
      \hline
      \multirow{5}{*}{$\tau=0.00$} & \quad Avg. $E[R_{t+1}]$ & 102.8\% & 102.8\% & 102.8\% \\
      & \quad $\sigma(R_{t+1})$ & 3.944 & 3.941 & 3.938 \\
      & \quad Avg. $\bar{R_t}$ & 100.5\% & 100.5\% & 100.5\% \\
      & \quad Avg. eq. prem. $E[R_{t+1}] - \bar{R_t]}$ & 2.3\% & 2.3\% & 2.3\% \\
      & \quad Avg. Sharpe ratio $\frac{E[R_{t+1}] - \bar{R_t]}}{\sigma(R_{t+1})}$ & 0.584 & 0.583 & 0.583 \\
      \hline
      \multirow{5}{*}{$\tau=0.11$} & \quad Avg. $E[R_{t+1}]$ & 103.8\% & 103.8\% & 103.8\% \\
      & \quad $\sigma(R_{t+1})$ & 4.193 & 4.190 & 4.187 \\
      & \quad Avg. $\bar{R_t}$ & 101.3\% & 101.3\% & 101.3\% \\
      & \quad Avg. eq. prem. $E[R_{t+1}] - \bar{R_t]}$ & 2.5\% & 2.5\% & 2.5\% \\
      & \quad Avg. Sharpe ratio $\frac{E[R_{t+1}] - \bar{R_t]}}{\sigma(R_{t+1})}$ & 0.592 & 0.592 & 0.592 \\
      \hline
      \multirow{5}{*}{$\tau=0.20$} & \quad Avg. $E[R_{t+1}]$ & 104.7\% & 104.7\% & 104.7\% \\
      & \quad $\sigma(R_{t+1})$ & 4.404 & 4.401 & 4.399 \\
      & \quad Avg. $\bar{R_t}$ & 102.1\% & 102.1\% & 102.1\% \\
      & \quad Avg. eq. prem. $E[R_{t+1}] - \bar{R_t]}$ & 2.6\% & 2.6\% & 2.6\% \\
      & \quad Avg. Sharpe ratio $\frac{E[R_{t+1}] - \bar{R_t]}}{\sigma(R_{t+1})}$ & 0.594 & 0.593 & 0.593 \\
      \hline
      \multirow{5}{*}{$\tau=0.25$} & \quad Avg. $E[R_{t+1}]$ & 105.3\% & 105.3\% & 105.3\% \\
      & \quad $\sigma(R_{t+1})$ & 4.523 & 4.520 & 4.518 \\
      & \quad Avg. $\bar{R_t}$ & 102.6\% & 102.6\% & 102.6\% \\
      & \quad Avg. eq. prem. $E[R_{t+1}] - \bar{R_t]}$ & 2.7\% & 2.7\% & 2.7\% \\
      & \quad Avg. Sharpe ratio $\frac{E[R_{t+1}] - \bar{R_t]}}{\sigma(R_{t+1})}$ & 0.593 & 0.593 & 0.593 \\
      \hline\hline
    \end{tabular}
    \begin{tablenotes}
      \scriptsize{\item[]The gross risky one-period return on capital is $R_{t+1} = 1 + r_{t+1}$ and the average expected gross risky return (Avg. $E[R_{t+1}]$) is the average value of $R_{t+1}$ across simulations. The annualized gross risky one-period return is $(R_{t+1})^{1/30}$. The standard deviation of $R_{t+1}$ is just the standard deviation of its realized value across simulations. The average riskless gross return (Avg. $\bar{R}_t$) is the average value across simulations, where $\bar{R}_t=1+\bar{r}_t$.}
    \end{tablenotes}
    \end{threeparttable}
  \end{table}

  \clearpage


\section{Conclusion}\label{SecConclusion}


\end{spacing}

\bibliography{Evans2020}


\newpage
\renewcommand{\theequation}{T.\arabic{section}.\arabic{equation}}
                                                 % redefine the command that creates the section number
\renewcommand{\thesection}{T-\arabic{section}}   % redefine the command that creates the equation number

\setcounter{equation}{0}                         % reset counter
\setcounter{section}{0}                          % reset section number
\section*{TECHNICAL APPENDIX}

\setcounter{equation}{0}                         % reset counter
\section{Description of calibration}\label{SecTAppCalib}

  This section details our calibration of the parameter values listed in Table \ref{TabCalibr}. In our two-period-lived agent OG model, we assume that each period represents 30 years or, equivalently, a lifetime is 60 years. The model-period (30-year) discount factor $\beta$ is set to match the annual discount factor common in the RBC literature of $0.96$.
  \begin{equation}\label{EqTAppCalib_beta}
    \beta = (0.96)^{30}\approx 0.2939
  \end{equation}
  We set the coefficient of relative risk aversion at a midrange value of $\gamma=2$. This value lies in the midrange of values that have been used in the literature.\footnote{Estimates of the coefficient of relative risk aversion $\gamma$ mostly lie between 1 and 10. See \citet{MankiwZeldes:1991}, \citet{Blake:1996}, \citet{Campbell:1996}, \citet{Kocherlakota:1996}, \citet{BravConstantinidesGeczy:2002}, and \citet{MehraPrescott:1985}.} The capital share of income parameter is set to match the U.S. average $\alpha=0.35$, and the model-period (30-year) depreciation rate $\delta$ is set to match an annual depreciation rate of 5 percent.
  \begin{equation}\label{EqTAppCalib_delta}
    \delta = 1 - (1 - 0.05)^{30}\approx 0.7854
  \end{equation}

  The firms' production function in our model is the following,
  \begin{equation}\tag{\ref{EqModelFirmProdFunc}}
    Y_t = e^{z_t}K_t^\alpha L_t^{1-\alpha} \quad\forall t
  \end{equation}
  where labor $L_t$ is supplied inelastically and $z_t$ is current-period normally distributed component of total factor productivity. We assume that $z_t$ is an AR(1) process with normally distributed errors.
  \begin{equation}\tag{\ref{EqModelFirmZAR1}}
    \begin{split}
      z_t &= \rho z_{t-1} + (1-\rho)\mu + \ve_t \\
      &\text{where}\quad \rho\in[0,1), \quad\mu\geq 0, \quad\text{and}\quad \ve_t \sim N(0,\sigma)
    \end{split}
  \end{equation}
  This implies that the shock process $e^{z_t}$ is lognormally distributed $LN(\rho z_t + (1-\rho)\mu,\sigma)$. The RBC literature calibrates the parameters on the shock process \eqref{EqModelFirmZAR1} to $\rho=0.95$ and $\sigma = 0.4946$ for annual data.

  For data in which one period is 30 years, we have to recalculate the analogous $\tilde{\rho}$ and $\tilde{\sigma}$.
  \begin{equation*}\label{TAppCalEqZtpj}
    \begin{split}
      z_{t+1} &= \rho z_{t} + (1-\rho)\mu + \ve_{t+1} \\
      z_{t+2} &= \rho z_{t+1} + (1-\rho)\mu + \ve_{t+2} \\
              &= \rho^2 z_{t} + \rho(1-\rho)\mu + \rho\ve_{t+1} + (1-\rho)\mu + \ve_{t+2} \\
      z_{t+3} &= \rho z_{t+2} + (1-\rho)\mu + \ve_{t+3} \\
              &= \rho^3 z_{t} + \rho^2(1-\rho)\mu + \rho^2\ve_{t+1} + \rho(1-\rho)\mu + \rho\ve_{t+2} + (1-\rho)\mu + \ve_{t+3} \\
              &\vdots \\
      z_{t+j} &= \rho^{j}z_{t} + (1-\rho)\mu\sum_{s=1}^{j}\rho^{j-s} + \sum_{s=1}^{j}\rho^{j-s}\ve_{t+s}
    \end{split}
  \end{equation*}
  With one period equal to thirty years $j=30$, the shock process in our paper should be:
  \begin{equation}\label{TAppCalEqZ30}
    z_{t+30} = \rho^{30}z_{t} + (1-\rho)\mu\sum_{s=1}^{30}\rho^{30-s} + \sum_{s=1}^{30}\rho^{30-s}\ve_{t+s}
  \end{equation}
  Then the persistence parameter in our one-period-equals-thirty-years model should be $\tilde{\rho}=\rho^{30}\approx 0.2146$ and the unconditional mean should be $\tilde{\mu} = \mu\sum_{s=1}^{30}\rho^{30-s}=0$. Define $\tilde{\ve}_{t+30}\equiv\sum_{s=1}^{30}\rho^{30-s}\ve_{t+s}$ as the summation term on the right-hand-side of \eqref{TAppCalEqZ30}. Then $\tilde{\ve}_{t+30}$ is distributed:
  \begin{equation*}\label{TAppCalEqEps30dist}
    \tilde{\ve}_{t+30}\sim N\Biggl(0,\biggl[\sum_{s=1}^{30}\rho^{2(30-s)}\biggr]^\frac{1}{2}\sigma\Biggr)
  \end{equation*}
  Using this formula, the annual persistence parameter $\rho=0.95$, and the annual standard deviation parameter $\sigma=0.4946$, the implied thirty-year standard deviation is $\tilde{\sigma}\approx 1.5471$. So our shock process should be,
  \begin{equation*}\label{TAppCalEqZ30cal}
    z_t = \tilde{\rho}z_{t-1} + (1-\rho)\tilde{\mu} + \tilde{\ve}_t \quad\forall t \quad\text{where}\quad \tilde{\ve}\sim N(0,\tilde{\sigma})
  \end{equation*}
  where $\tilde{\rho}=0.2146$ and $\tilde{\sigma}=1.5471$. We arbitrarily choose $\mu=\tilde{\mu}=0$. However, we could have also chosen $\mu$ and the corresponding $\tilde{\mu}$ to his a median wage target.

  Lastly, we set the size of the promised transfer $\bar{H}$ to be 32 percent of the median real wage. This level of transfers is meant to approximately match the average per capita real transfers in the United States to the average real wage in recent years. We get the median real wage by simulating a time series of the economy until it hits the shut down point, and we do this for 3,000 simulated time series. We take the median wage from those simulations. In order to reduce the effect of the initial values on the median, we take the simulation that lasted the longest number of periods before shutting down and remove the first 10 percent of the longest simulation's periods from each simulation for the calculation of the median.


\newpage
\setcounter{equation}{0}                         % reset counter
\section{Truncated Normal Distribution}\label{SecTAppTrNdist}

  Put a description of the properties of the truncated normal distribution here and how it interacts with the total factor productivity process.


\newpage
\section{Comments and Notes}\label{TAppCommentsNotes}

  \begin{itemize}
    \item Interesting papers on debt and rare events: \citet{RebeloEtAl:2019}, \citet{ReinhartEtAl:2015}
    \item Equity premium puzzle explanations
    \begin{itemize}
      \item General: \citet{DeLongMagin:2009}, \citet{FarhiGourio:2019}
      \item Prospect theory by Kahneman and Tversky
      \item the role of personal debt
      \item the importance of credit risk and liquidity: \citet{Gourio:2013}
      \item the impact of government regulation
      \item consideration of taxes
      \item rare events/disasters: see references in \citet{TsaiWachter:2015}, including \citet{Barro:2009}, \citet{Gourio:2012}
    \end{itemize}
    \item Our current calibration does not match the NBER paper because I am not sure I like the way we calculated the median (see last paragraph of Technical Appendix \ref{SecTAppCalib}.).
    \item Rick thinks we should use language of ``fiscal limit'', and not use the more extreme terms such as ``shutdown" or ``game over".
    \item Does Ricardian equivalence hold in this model? Agents have rational expectations, and they smooth consumption. But the government runs a balanced budget in each period. My intuition is that this is a Ricardian model because agents expect that the government will eventually have to default on its promised transfer $\bar{H}$.
    \item In describing the transfer program, justify lump sum transfers as approximating a degree of fiscal inertia or fiscal stickiness. Policy stickiness could either speed up the  expected time until the economy hits its fiscal limit, or it could delay policy responses after hitting the limit which make outcomes worse. Papers that incorporate policy stickiness into stochastic OLG models are \citet{AuerbachHassett:1992,AuerbachHassett:2001,AuerbachHassett:2002,AuerbachHassett:2007} and \citet{HassettMetcalf:1999}. \citet{AlesinaDrazen:1991} discuss the foundations of fiscal stickiness.
    \item List of papers that focus ``on important intergenerational and distributional consequences of fiscal stress and fiscal limits": \citet{AuerbachKotlikoff:1987}, \citet{KotlikoffSmettersWalliser:1998a,KotlikoffSmettersWalliser:1998b,KotlikoffSmettersWalliser:2007}, \citet{Imrohoroglu2Joines:1995,Imrohoroglu2Joines:1999}, \citet{HuggettVentura:1999}, \citet{CooleySoares:1999}, \citet{DeNardiImrohorogluSargent:1999}, \citet{AltigAuerbachKotlikoffSmettersWalliser:2001}, \citet{SmettersWalliser:2004}, and \citet{NishiyamaSmetters:2007}.
  \end{itemize}


\end{document}
